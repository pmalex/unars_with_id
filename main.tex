\documentclass[11pt,twoside,draft
]{article}
\usepackage{amsmath,amsfonts,amssymb,amsthm,indentfirst,enumerate,textcomp}
\usepackage[utf8]{inputenc}
\usepackage{chebsb}
\usepackage[russian]{babel}
\usepackage{indentfirst, array}
\usepackage{amscd,latexsym}
\usepackage{mathrsfs}
%\usepackage[ruled, linesnumbered]{algorithm2e}
\usepackage{tabularx}
\usepackage{multirow}

\usepackage{graphicx}
\usepackage{textcase}% Оформление страниц


\label{beg}
% заглавие стати и аннотация

\levkolonttl{%левый колонтитул - авторы
И.~И.~Иванов}
\prvkolonttl{%правый колонтитул - сокращенное название статьи
О квадратах и кубиках \ldots}

\newcommand{\god}{2025}

\UDK{% УДК статьи
517}

\DOI{
10.22405/2226-8383-\god-\tom-\iss-\pageref{beg}-\pageref{end}
}

\title
{%название статьи на русском языке с указанием источника финансирования при необходимости
О квадратах и кубиках в множествах конечного поля\footnote{Исследование выполнено за счет гранта Российского научного фонда (проект ХХ-ХХ-ХХХХХ).}}
{%название статьи на английском языке
On the squares and cubes in the set of finite fields}

\author
{%авторы статьи
И. И. Иванов}
{%авторы статьи
I. I. Ivanov}

\Cite{
И. И. Иванов. О квадратах и кубиках в множествах конечного поля // Чебышевcкий сборник, \god, т.~\tom, вып.~\iss, с.~\pageref{beg}--\pageref{end}.
}
{Ivanov, I. I. \god, ``On the squares and cubes in the set of finite fields''\,, {\it Che\-by\-shev\-skii sbornik}, vol.~\tom, no.~\iss, pp.~\pageref{beg}--\pageref{end}.
}

\info
{%авторы статьи на русском языке
\noindent {\bf Иванов Иван Иванович}~--- ученая степень, организация (г. Город).

\noindent
\emph{e-mail: }


}
{%авторы статьи на английском языке
\noindent {\bf Ivanov Ivan Ivanovich}~--- degree, organization (City).

\noindent
\emph{e-mail: }

}

\Abstract
{%Аннотация статьи на русском языке 150-250 слов с учетом ключевых слов
В статье...
}
{%Аннотация статьи на английском языке
In paper...
}

\keywords
{%ключевые слова на русском языке
конечные поля, квадраты, суммы.
}
{%ключевые слова на английском языке
finite fields, squares, sums.
}

%число наименований в библиографии
\Bibliography{18 названий.}{18 titles.}

\begin{document}

%генерация заглавия статьи
\maketitle

\enmaketitle

\section{Введение}

\section{Основной текст статьи}

%Пример оформления изображений
%Image design example

\begin{figure}[ht!]
  \centering
  \includegraphics[scale=1]{Ivanov/img1.png} %Вместо scale можно использовать [width=1\linewidth] или [width=150mm]
  \caption{Подпись к рисунку}
  \label{formula 1} %если нужен ярлык
\end{figure}

\section{Заключение}

%библиография по ГОСТу
\begin{thebibliography}{99}
	
	\bibitem{Ivanov_1}
	Banks~W.\,D., Conflitti~A., Shparlinski~I.\,E. Character sums over integers with restricted $g$-ary digits // Illinois J. Math. 2002. Vol. 46, №3. P. 819--836.
		

\end{thebibliography}


%библиография по Гарвардскому стандарту
\begin{engbibliography}{99}
	
	\bibitem{en_Ivanov_1}
	Banks, W.\,D., Conflitti, A. \& Shparlinski, I.\,E. 2002, ``Character sums over integers with restricted $g$-ary digits'', \textit{Illinois J. Math.}, vol. 46, no. 3, pp. 819--836.

\end{engbibliography}

\label{end}

\end{document} 