\documentclass[11pt,twoside,draft
]{article}
\usepackage{amsmath,amsfonts,amssymb,amsthm,indentfirst,enumerate,textcomp}
\usepackage[utf8]{inputenc}
\usepackage{chebsb}
\usepackage[russian]{babel}
\usepackage{indentfirst, array}
\usepackage{amscd,latexsym}
\usepackage{mathrsfs}
%\usepackage[ruled, linesnumbered]{algorithm2e}
\usepackage{tabularx}
\usepackage{multirow}

\usepackage{graphicx}
\usepackage{textcase}% Оформление страниц


\label{beg}
% заглавие стати и аннотация

\levkolonttl{%левый колонтитул - авторы
И.~Б.~Кожухов, А.~М.~Пряничников}
\prvkolonttl{%правый колонтитул - сокращенное название статьи
Об унарах с тождествами в решётке конгруэнций, II}

\newcommand{\god}{2025}

\UDK{% УДК статьи
512.567.5 + 512.579}

\DOI{
10.22405/2226-8383-\god-\tom-\iss-\pageref{beg}-\pageref{end}
}

\title
{%название статьи на русском языке с указанием источника финансирования при необходимости
Об унарах с тождествами в решётке конгруэнций, II}
{%название статьи на английском языке
On unars with identities in congruence lattice, II}

\author
{%авторы статьи
И. Б. Кожухов, А. М. Пряничников}
{%авторы статьи
I. B. Kozhukhov, A. M. Pryanichnikov}

\Cite{
И. Б. Кожухов, А. М. Пряничников. Об унарах с тождествами в решётке конгруэнций, II // Чебышевcкий сборник, \god, т.~\tom, вып.~\iss, с.~\pageref{beg}--\pageref{end}.
}
{Kozhukhov, I. B., Pryanichnikov, A. M. \god, ``On unars with identities in congruence lattice, II''\,, {\it Che\-by\-shev\-skii sbornik}, vol.~\tom, no.~\iss, pp.~\pageref{beg}--\pageref{end}.
}

\info
{%авторы статьи на русском языке
\noindent {\bf Кожухов Игорь Борисович}~--- д.ф.-м.н., Нац. исслед. университет МИЭТ; механико-математический факультет МГУ; Гос. академия нар. хоз-ва и гос. службы (г. Москва).

\noindent
\emph{e-mail: kozhuhov\_i\_b@mail.ru}

\noindent {\bf Пряничников Алексей Михайлович}~--- механико-математический факультет МГУ, АО <<Концерн Гранит>> (г. Москва).

\noindent
\emph{e-mail: genary@ya.ru}


}
{%авторы статьи на английском языке
\noindent {\bf Kozhukhov Igor Borisovich}~--- Dr. Phys.-Math. Sci., Nat. Res. Univ. MIET; Fac. of Mech. and Math. of Moscow State Univ.; Russian Presidental Academy of Nat. Econom. and Public Admin., (Moscow).

\noindent
\emph{e-mail: kozhuhov\_i\_b@mail.ru}

\noindent {\bf Pryanichnikov Alexey Mikhailovich}~--- Fac. of Mech. and Math. of Moscow State Univ., <<Concern Granit>> JSC, (Moscow).

\noindent
\emph{e-mail: genary@ya.ru}

}

\Abstract
{%Аннотация статьи на русском языке 150-250 слов с учетом ключевых слов
В статье доказано, что если решётка конгруэнций унара удовлетворяет нетривиальному решёточному тождеству, то унар является гомоморфным образом копроизведения конечного числа прямых и лучей.
}
{%Аннотация статьи на английском языке
In paper we prove that if a congruence lattice of an unar holds a nontrivial lattice identity, then the unar is a homomorphic image of a coproduct of finite numbers of lines and rays.
}

\keywords
{%ключевые слова на русском языке
полигон над полугруппой, унар, решётка конгруэнций, тождество.
}
{%ключевые слова на английском языке
act over semigroup, unar, congruence lattice, identity.
}

%число наименований в библиографии
\Bibliography{15 названий.}{15 titles.}

\def\Con{\operatorname{Con}}
\def\Eq{\operatorname{Eq}}

\begin{document}

%генерация заглавия статьи
\maketitle

\enmaketitle

\section{Введение}

Решётка конгруэнций $\Con \mathbf{A}$ универсальной алгебры $\mathbf{A}$ — важная характеристика этой алгебры.
Это полная решётка с наименьшим элементом $\Delta = \{ (a,a) \mid a \in \mathbf{A} \}$ и наибольшим элементом $\nabla = \mathbf{A} \times \mathbf{A}$, причём $\Con \mathbf{A}$ является полной подрешёткой решётки $\Eq \mathbf{A}$ всех отношений эквивалентности на множестве $\mathbf{A}$.
Изучение алгебр с условиями на решётку конгруэнций — активно развивающееся и богатое содержанием направление общей алгебры.
Сюда относятся условия максимальности и минимальности, приводящие к понятиям артиновых и нётеровых алгебр, условие простоты (когда $\Con \mathbf{A} = \{ \Delta, \nabla \}$), антипростоты ($\Con \mathbf{A} = \Eq \mathbf{A}$), дистрибутивности или модулярности (т.е. решётка $\Con \mathbf{A}$ дистрибутивна или модулярна), подпрямой неразложимости и т.д.

\textit{Полигон над полугруппой} — это множество $X$, на котором действует полугруппа $S$, т.е. определено отображение $X \times S \to X$, $(x,s) \mapsto xs$ удовлетворяющее условию $x(st) = (xs)t$ при $x \in X$, $s,t \in S$ (см.~\cite{Kilp_1}).
Полигон над полугруппой является алгебраической моделью \textit{автомата}.
Кроме того, понятие полигона фактически совпадает с понятием \textit{унарной алгебры}.

Подмножество $Y \subseteq X$ называется \textit{подполигоном}, если $YS \subseteq Y$.
Элемент $z \in X$ называется \textit{нулём}, если $zs = z$ при всех $s \in S$.
Нулей у полигона, в отличие от полугруппы, может быть сколько угодно (в полугруппе нуль, если он существует, единственен).
\textit{Решётку конгруэнций} полигона $X$ над полугруппой $S$ мы будем обозначать $\Con_S X$ или просто $\Con X$.

Пусть $X$ — полигон над полугруппой $S$ и $Y$ — его подполигон.
Положим $ \rho_Y = (Y \times Y) \cup \Delta_X $.
Это конгруэнция полигона $X$, называемая \textit{конгруэнцией Риса}, соответствующей подполигону $Y$.
Нетрудно видеть, что для любого подполигона $Y$, у конгруэнции Риса $\rho_Y$ один из классов есть множество $Y$, а другие классы одноэлементны.
Для фактор-полигона $X/\rho_Y$ мы будем также использовать краткую запись: $X/Y$.

Пусть $X$ — полигон над полугруппой $S$ и $X_i$ ($i \in I$) — его подполигоны, причём $X = \bigcup_{i \in I} X_{i}$ и $X_{i} \cap X_{j} = \varnothing$ при $i \neq j$.
В этом случае мы называем полигон $X$ \textit{копроизведением} полигонов $X_i$ и пишем $X = \bigsqcup_{i \in I} X_{i}$.
В работах по унарам копроизведение называют обычно прямой суммой.

Для полугруппы $S$ полагаем $S^1 = S \cup \{ 1 \}$.
Здесь $S^1$ — полугруппа, полученная присоединением к $S$ внешним образом единицы (при этом считаем, что $s \cdot 1 = 1 \cdot s = s$ для каждого $s \in S^1$).
Если $X$ — полигон над $S$, то его можно сделать полигоном над полугруппой $S^1$, положив $x \cdot 1 = x$ при $x \in X$.

На полигоне $X$ над полугруппой $S$ можно ввести отношение \textit{квазипорядка} $\leqslant$ (рефлексивное и транзитивное отношение), полагая $x \leqslant y \Leftrightarrow x \in yS^1$.
Полигон $X$ называется \textit{связным}, если для любых $x,y \in X$ существуют элементы $x_1,\ldots,x_{2n} \in X$ такие, что  $ x \leqslant x_1 \geqslant x_2 \leqslant \ldots \geqslant x_{2n} \leqslant y$.
Нетрудно видеть, что любой полигон $X$ является копроизведением связных подполигонов $X_i$ (\textit{компонент связности}).

\textit{Унар}, т.е. множество $X$ с одной унарной операцией $f: X \to X$, можно рассматривать как полигон над свободной циклической полугруппой $S = \{ a,a^{2},a^{3},\ldots\}$, при этом $x \cdot a = f(x)$ для $x \in X$.

Алгебрам $\mathbf{A}$, у которых решётка конгруэнций $\Con \mathbf{A}$ является дистрибутивной или модулярной, посвящено значительное число работ разных авторов. 
Дистрибутивные и модулярные полигоны над полугруппами изучались в~\cite{Ptakhov_2}, полное описание таких полигонов над полугруппами левых или правых нулей получено в~\cite{Khaliullina_3}. 
В работе~\cite{Egorova_4} были описаны дистрибутивные и модулярные унары.
Дистрибутивность решётки $\Con \mathbf{A}$ означает выполнение в ней тождества $ (x \vee y) \wedge z = (x \wedge z) \vee (y \wedge z) $, модулярность задаётся тождеством $ (x \vee y) \wedge (x \vee z) = x \vee (z \wedge (x \vee y)) $.
Естественным представляется изучение алгебр $\mathbf{A}$, у которых решётка $\Con \mathbf{A}$ удовлетворяет какому-нибудь нетривиальному решёточному тождеству.
Данное условие является \textit{условием конечности}, так как любая конечная алгебра ему удовлетворяет (это следует из того, что всякая конечная решётка удовлетворяет нетривиальному тождеству — см. лемму 3).
Многообразия алгебр, у которых решётки конгруэнций всех алгебр удовлетворяют одному и тому же нетривиальному тождеству, посвящены монография~\cite{Kearnes_5} и диссертация~\cite{Nation_6}.
В работе~\cite{Repnitsky_7} были описаны коммутативные полугруппы, у которых решётки подполугрупп удовлетворяют нетривиальному решёточному тождеству.
Цель данной работы — найти необходимое условие того, что решётка конгруэнций унара удовлетворяет нетривиальному тождеству.
Является ли это условие достаточным, авторам неизвестно.

Работа является продолжением работы~\cite{Kozhukhov_8}.
Основные определения мы здесь приведём, за доказательствами отошлём читателя к работе~\cite{Kozhukhov_8}.
Отметим, что заключительное утверждение, сформулированное в последнем абзаце работы~\cite{Kozhukhov_8}, требует корректировки: вместо «копроизведения конечного числа прямых» следовало написать «копроизведения конечного числа прямых и лучей».

Основные сведения из универсальной алгебры можно найти в~\cite{Kohn_9}, из теории решёток — в~\cite{Gretzer_10}, теории полугрупп — в~\cite{Clifford_11}, полигонов над полугруппами — в~\cite{Kilp_1}, унаров — в~\cite{Jakubikova_12}, многообразий решёток — в~\cite{Jipsen_13}.


\section{Основной текст статьи}

%Пример оформления изображений
%Image design example

\begin{figure}[ht!]
	\centering
	\includegraphics[scale=1]{Ivanov/img1.png} %Вместо scale можно использовать [width=1\linewidth] или [width=150mm]
	\caption{Подпись к рисунку}
	\label{formula 1} %если нужен ярлык
\end{figure}

\section{Заключение}

%библиография по ГОСТу
\begin{thebibliography}{99}
	
	\bibitem{Kilp_1}
	Kilp~M., Knauer~U., Mikhalev~A.\,V. Monoids, acts and categories // Berlin - N.Y., de Gruyter, 2000. 529 P.
	
	\bibitem{Ptakhov_2}
	Птахов~Д.\,О., Степанова~А.\,А. Решётки конгруэнций полигонов // Дальневост. матем. журн. 2013. Т.~13, №1. С.~107--115.
	
	\bibitem{Khaliullina_3}
	Халиуллина~А.\,Р. Условия модулярности решётки конгруэнций полигона над полугруппой правых или левых нулей // Дальневост. матем. журн. 2015. Т.~15, №~1. С.~102--120.
	
	\bibitem{Egorova_4}
	Егорова~Д.\,П. Структура конгруэнций унарной алгебры // Межвузовский научный сборник «Упорядоченные множества и решётки». г. Саратов: Издательство Саратовского университа. 1978. Вып.~5. С.~11--43.
	
	\bibitem{Kearnes_5}
	Kearnes~K.\,A., Kiss~E.\,W. The shape of congruence lattices // Memoirs of the American Mathematical Society. 2013. Vol.~222. 169 P.
	
	\bibitem{Nation_6}
	Nation~J.\,B. Varieties of algebras whose congruence lattices satisfy lattice identities (Thesis) // Pasadena: California Institute of Technology. 1973. 63 P.
	
	\bibitem{Repnitsky_7}
	Репницкий~В.\,Б., Кацман~С.\,И. Коммутативные полугруппы, решётка подполугрупп которых удовлетворяет нетривиальному тождеству // Математический сборник. 1988. Т.~137(179), №~4(12). С.~462--482.
	
	\bibitem{Kozhukhov_8}
	Кожухов~И.\,Б., Пряничников~А.\,М. Об унарах с тождествами в решётке конгруэнций // Материалы VI международной научно-технической конференции СИТОНИ-2019. Донецк, 2019. С.~64--69.
	
	\bibitem{Kohn_9}
	Кон~П.\,М. Универсальная алгебра // М.: Мир, 1968. 359~С.
	
	\bibitem{Gretzer_10}
	Гретцер~Г. Общая теория решёток // М.: Мир, 1982. 454~С.
	
	\bibitem{Clifford_11}
	Клиффорд~А., Престон~Г. Алгебраическая теория полугрупп // М.: Мир, 1972. Т.~1. 286~с.; Т.~2. 423~С.
	
	\bibitem{Jakubikova_12}
	Jakubiková-Studenovská~D., Pócs~J. Monounary algebras // Košice: UPJS, 2009. 301~P.
	
	\bibitem{Jipsen_13}
	Jipsen~P., Rose~H. Variety of lattices // Lecture Notes in Mathematics, 1992. Vol.~1533. 166~P.
	
	\bibitem{Burris_14}
	Burris~S., Sankappanavar~H.\,P. A course in universal algebra // Springer New York, 1981, Vol.~78, 276~P.
	
\end{thebibliography}


%библиография по Гарвардскому стандарту
\begin{engbibliography}{99}
	
	\bibitem{en_Kilp_1}
	Kilp, M., Knauer, U. \& Mikhalev, A.\,V. 2000, ``Monoids, acts and categories``, \textit{Berlin; New York: de Gruyter}, 546 pp.
	
	\bibitem{en_Ptakhov_2}
	Ptahov, D.\,O. Stepanova, A.\,A. 2013, ``Congruence lattice of S-acts``, \textit{Far Eastern Mathematical Journal}, vol. 13, no. 1, pp. 107--115.
	
	\bibitem{en_Khaliullina_3}
	Khaliullina, A.R. 2015, ``Modularity conditions of the lattice of congruences of acts over right or left zero semigroups``, \textit{Far Eastern Mathematical Journal}, vol. 15, no. 1, pp. 102--120.
	
	\bibitem{en_Egorova_4}
	Egorova, D.\,P. 1978, ``The structure of congruences of unary algebra``, \textit{Interuniversity scientific collection "Ordered sets and lattices"}, Saratov: Saratov University Press, issue 5, pp. 11--4. (in Russian)
	
	\bibitem{en_Kearnes_5}
	Kearnes, K.\,A., Kiss, E.\,W. 2013, ``The shape of congruence lattices``, \textit{Memoirs of the American Mathematical Society}, vol. 222, 169 pp.
	
	\bibitem{en_Nation_6}
	Nation, J.\,B. 1973, ``Varieties of algebras whose congruence lattices satisfy lattice identities``, \textit{PhD thesis, California Institute of Technology}, 63 pp.
	
	\bibitem{en_Repnitsky_7}
	Repnitsky, V.\,B., Katsman, S.\,I. 1990, ``Commutative semigroups with lattice of subsemigroups satisfies a nontrivial identity``, \textit{Mathematics of the USSR-Sbornik}, vol. 65, no. 2, pp. 465--485.
	
	\bibitem{en_Kozhukhov_8}
	Kozhukhov, I.\,B., Pryanichnikov, A.\,M. 2019, ``On unars with identities in congruence lattice``', \textit{Proceedings of the VI International Scientific and Technical Conference SITONI-2019}, Donetsk, pp. 64--69. (in Russian)
	
	\bibitem{en_Kohn_9}
	Kohn, P.\,M. 1981, ``Universal algebra``, \textit{Springer Dordrecht}, 412 pp.
	
	\bibitem{en_Gretzer_10}
	Grätzer, G. 2011, \textit{Lattice Theory: Foundation}, Birkhäuser Basel, 614 pp.
	
	\bibitem{en_Clifford_11}
	Clifford, A.\,H., Preston, G.\,B. 1961--1967, ``The algebraic theory of semigroups``, \textit{American Mathematical Society. Mathematical Surveys}, Providence, Rhode Island, no. 7, vol. I-II, 244 pp. and 350 pp.
	
	\bibitem{en_Jakubikova_12}
	Jakubiková-Studenovská, D., Pócs, J. 2009, ``Monounary algebras``, \textit{UPJS}, Košice, 301 pp.
	
	\bibitem{en_Jipsen_13}
	Jipsen, P., Rose, H. 1992, Variety of lattices, \textit{Lecture Notes in Mathematics}, Springer Berlin, vol. 1533, 166 pp.
	
	\bibitem{en_Burris_14}
	Burris, S., Sankappanavar, H.P. 1981, A course in universal algebra, \textit{Springer New York}, vol. 78, 276 pp.
	
\end{engbibliography}

\label{end}

\end{document}